%% This file is to be used as a template for your submission. 
%% Rename this file and replace the text with the text of 
%% your manuscript.
%%
%% The standard LaTeX document class "article" is recommended. 
%% Use options letterpaper and 12pt.
\documentclass[letterpaper,12pt]{article}

%% This is the recommended preamble for your document.

%% Load De Gruyter specific settings 
\usepackage{dgjournal}          

%% The mathptmx package is recommended for Times compatible math symbols.
%% Use mtpro2 or mathtime instead of mathptmx if you have the commercially
%% available MathTime fonts.
%% Other options are txfonts (free) or belleek (free) or TM-Math (commercial)
\usepackage{mathptmx}

% Big maths font
%\everymath{\displaystyle}
\everymath=\expandafter{\the\everymath\displaystyle}

%% Use the graphics package to include figures
\usepackage{graphicx}
\usepackage{float}

%% Use natbib with these recommended options
\usepackage[authoryear,comma,longnamesfirst,sectionbib]{natbib} 
\usepackage{url}

%% Start your document body here
\begin{document}

%% Do NOT include any fronmatter information; including the title, author names,
%% institutes, acknowledgments and title footnotes (author information, funding
%% sources, etc.). Start the document with the first section or paragraph of
%% the article.

\begin{abstract}

Tennis is a hugely popular sport around the globe. Its fans include
not only spectators attracted by the excitement of the game but also
speculators attracted by the possibility of making large financial
gains.  Indeed, every on-court development drives the movement of
several in-play financial markets.

As in other financial markets, sophisticated traders make use of
quantitative models in an attempt to gain an advantage over their
peers. For a tennis singles match, the models are usually based on
hierarchical Markov chains and typically yield the probability of each
player winning the match as a function of three parameters: the
current score, and the probability of each player winning a point on
their serve.

These models, however, ignore the potential impact of retirement risk
due to injuries, which can drastically alter in-play betting odds~--
even during the course of a single point. The exact nature of the
impact on a given market depends on the payout policy of that market
in the case of player retirement.

This paper proposes novel quantitative models which reflect the
in-play evolution of retirement risk due to injury and the
corresponding impacts on in-play financial markets based on match
outcome under different retirement payout policies. Injuries are
assumed to occur according to a Bernoulli process, with a magnitude
sampled from a truncated exponential distribution. Retirement risk is
propagated from one point to another subject to a damping parameter
that reflects a player's ability to ``run off'' an injury.

In terms of practical deployment, a challenging problem is to
correctly parameterise the models given the availability of a current
score feed and odds data in two match outcome markets with different
retirement policies. Artificially generated match data shows that it
is possible to infer model parameters effectively. We also find that
we are able to follow the progression of Betfair in-play tennis
markets for a number of real-life matches to a good degree of accuracy
and can provide a value for the retirement risk of a given player at
any point.  It is also possible to predict the evolution of
match outcome markets that feature other retirement policies.

\end{abstract}

\end{document}